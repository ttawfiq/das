
\documentclass[a4paper,12pt]{article}
\usepackage{noweb}%\pagestyle{noweb}
\noweboptions{noxref,smallcode}
\addtolength{\topmargin}{-0.5in}
\setlength{\textheight}{9in}
\setlength{\textwidth}{6.9in}
\setlength{\oddsidemargin}{-.3in}
\setlength{\evensidemargin}{-.1in}
\begin{document}

\def\nwendcode{\endtrivlist \endgroup}
\let\nwdocspar=\par

\date{August 9th 2006}
\title{An Assembler}

\author{Daniel Bastos\\
        \tt{dbastos@toledo.com}}

\maketitle

\abstract{This is an educational assembler studied when reading "The
AWK Programming Language" by Brian Kernighan. It's very simple, but
quite educational. It's written in AWK, the language the book
describes.}

\setlength{\parskip}{7pt}
\setlength{\parindent}{0pt}

\section{Introduction} We're writing an assembler that implements a
simple assembly language for a simple hypothetical computer. The
computer has one register --- which we often make reference to by using
the symbol [[%r]] ---, ten instructions and a memory capable of a
thousand words. Each word is composed of 5 bits, each bit is represented
by a digit. If a word is a machine instruction, then the first two bits
encode the operation and the last three digits are the memory
address. The memory addresses range from 0 to 999.

Each program is a sequence of statements, where each statement is
separated by a new line character. Each statement is composed of three
fields: a label, an operation and an operand. Spaces or tabs, in any
amount, separate the fields. Comments start anywhere after the character
[[#]] and end once a new line is found.

We implement the following instructions: [[const]], [[get]], [[put]],
[[ld]], [[st]], [[add]], [[sub]], [[jp]], [[jz]], [[j]], [[halt]]. These
instructions, respectively, have the operation codes 00, 01, 02, 03, 04,
05, 06, 07, 08, 09, 10. The keyword [[const]] isn't precisely an
instruction. It is an assembler directive to define a constant.
%
%\begin{verbatim}
%label const value             # defines a constant
%get                           # read input into register
%put                           # writes register data to output               
%ld mem                        # loads mem's data into register
%st mem                        # saves register's data into mem
%add mem                       # register + *mem
%sub mem                       # register - *mem
%jp mem                        # goto mem if %r is positive
%jz mem                        # goto mem if %r is zero
%j mem                         # goto mem
%halt                          # exit(0)
%\end{verbatim}

The instruction [[get]] reads a number from the input into the
register. The instruction [[put]] writes the data from the register to
the output. The instruction [[ld mem]] loads the register with the data
in memory location [[mem]]. The instruction [[st]] saves the data
contained in the register at the memory location [[mem]]. The
instruction [[add]] adds the number from the location [[mem]] to the
number in the register. The instruction [[sub]] substracts the number
from the location [[mem]] to the number in the register. The instruction
[[jp]] jumps execution to [[mem]] if the number in the register is
positive. The instruction [[jz]] jumps execution to [[mem]] if the
number in the register is zero. The instruction [[j]] jumps execution to
[[mem]] unconditionately.
